\documentclass{article}
\usepackage[margin=1in]{geometry}
\usepackage[toc,page]{appendix}
\renewcommand{\appendixpagename}{}
\usepackage{subcaption}
\usepackage{graphicx}
\usepackage{natbib}
\usepackage{hyperref}
\usepackage[utf8]{inputenc}
\usepackage[T1]{fontenc}
\usepackage[french]{babel}
\usepackage{caption}
\usepackage{amsmath}
\usepackage[export]{adjustbox}
\usepackage[justification=centering]{caption}
\usepackage{listings}
\usepackage{multirow}
\usepackage{multicol}


%----------En-tête et pieds de page
\usepackage{blindtext}
\usepackage{lastpage}
\usepackage{fancyhdr}
\pagestyle{fancy}

%-------------------------
\usepackage{float}

%-------------chapitre non numéroté------
\newcommand{\nonumsection}[1]{
    \addcontentsline{toc}{section}{#1}
    \section*{#1}
}

%---------pour enlever les cadres moches des ref et de la tab
\hypersetup{
    linktoc=all,     %set to all if you want both sections and subsections linked
    colorlinks,
    citecolor=black,
    filecolor=black,
    linkcolor=black,
    urlcolor=black
}

%--------biblio numéroté dans l'ordre d'apparition
\usepackage[backend=biber,safeinputenc, sorting=none]{biblatex}
\addbibresource{references.bib}

% Please add the following required packages to your document preamble:
\usepackage{multirow}
\usepackage[table,xcdraw,equation,dvipsnames]{xcolor}
% dvipsnames donne accès à plus de couleurs? Pas sûr que equation soit utile

%--------- faire un glossaire-------------------
\usepackage{glossaries}
\makeglossaries

% Définir des couleur
\definecolor{Red_univ}{RGB}{229, 68, 47}
\definecolor{Blue_univ}{RGB}{29, 35, 67}
%%%%%%%%%%%%%%%%%%%%%%%%%%%%%%%%%%%%%%%%%%%%%%%%%%%%%%%%%%%%%%%%%%%%%%%%%%
%%%%%%%%%%%%%%%%%%%%%%%%%%%%%%%%%%%%%%%%%%%%%%%%%%%%%%%%%%%%%%%%%%%%%%%%%%
%%%%%%%%%%%%%%%%%%%%%%%%%%%%%%%%%%%%%%%%%%%%%%%%%%%%%%%%%%%%%%%%%%%%%%%%%%
\begin{document}

%-------------Page de garde----------------------

%-------------page de garde------
\begin{titlepage}
\begin{center}

\begingroup

\begin{figure}[htpb!]
    \centering
    \begin{tabular}{c c}
        \includegraphics[width=0.48\textwidth]{Images/0-Logos/logo_laum.png} &
        % \hspace{1.5cm}
        \includegraphics[width=0.48\textwidth]{Images/0-Logos/logo_lium.png} \\ 
    \end{tabular}
\end{figure}

\endgroup

\vspace{0.5cm}
\rule{\linewidth}{0.3mm}
\vspace{0.5cm}

\LARGE{\textsc{Prédire les descripteurs audio mesurés sur MAD dynamique à l'aide des indicateurs objectifs mesurés sur bancs de raideur}}

\vspace{0.5cm}
\rule{\linewidth}{0.3mm}

\vspace{1cm}

\LARGE{ \textsc{Camille Urban}} \\ [0.7cm]

\Large {Stage de fin d'études de cycle ingénieur à l'\textit{ENSIM} et de Master 2 de Recherche en Acoustique Appliquée à \textit{Le Mans Université}} \\[0.7cm]

11 mars - 27 sptembre 2024 \\ [0.7cm]

\begin{figure}[htpb!]
    \centering
    \includegraphics[width=0.48\textwidth]{Images/0-Logos/LOGO-ensim-couleur-HD.png} 
\end{figure}



\begin{multicols}{2}
\raggedright
\textbf{\textit{Tuteur université :}} \\
\textsc{Thibault Vicente} \\
    thibault.vicente@univ-lemans.fr \\


\columnbreak

\raggedleft
\textbf{\textit{Tuteur et tutrices entreprise :}}\\
\textsc{Bruno Gazengel} \\
bruno.gazengel@univ-lemans.fr \\
\textsc{Marie Tahon} \\
marie.tahon@univ-lemans.fr \\
% \textsc{Amélie Gaillard} \\
% amelie.gaillard@univ-lemans.fr 


\end{multicols}


\vfill

% \today\normalsize{}
\end{center}

\end{titlepage}

%-------------------tête de page-----------------
\fancyhf{}
\fancyfoot[C]{\thepage \hspace{1pt} | \pageref{LastPage}}


\pagenumbering{gobble} % oulala on ne veut pas encore de numéros de page
\newpage
\strut 
\newpage

%-------------RESUME----------------------------
\fancyhead[C]{Résumé}
\vfill
\begin{center}
\nonumsection{Résumé}
\end{center}
\vfill
% Résumé
% un résumé (<160 mots) posant BRIEVEMENT le contexte du
% stage, la façon dont il a été abordé et les principaux résultats obtenus. De plus, vous ajouterez à la suite les mots clés renseignés dans la fiche du stage

Ce stage s'inscrit dans le cadre d'une thèse en cours intitulée ... et menée par Amélie Gaillard. L'anche est un petit bout de roseau taillé qui est mis en vibration par le souffle du musicien, créant alors un son. Toute fois, pour deux anches vendues comme identiques par le fabricant, des musiciens à anches simples, comme les saxophonistes et les clarinettistes, sont capables de percevoir une différence et de conclure sur la qualité de l'anche. L'objectif final est de comprendre ce qui permet de qualifier une anche comme bonne ou mauvaise en se basant sur les caractéristiques mécaniques de cette dernière. Si la thèse d'Amélie Gaillard vient mettre en lien des études physiques et perceptives, ma contribution a pour but d'apporter des informations complémentaires via l'utilisation de modèles intelligence artificielle. Ces derniers viseront à mettre en lien des descripteurs audio mesurés à l'aide d'un musicien artificiel et des indicateurs mécaniques obtenus par des mesures sur bancs de raideurs. 

\vfill
\hline
\vspace{0.5cm}
\begin{center}
\nonumsection{Abstract} 
\end{center}
\vfill
% Abstract
% Toujours en deuxième page, ajoutez le même résumé et les mots-clés en anglais


This internship is part of an ongoing thesis entitled ... led by Amélie Gaillard. The reed is a small piece of cut cane wich is set in vibration by the musician's breath, creating a sound. However, for two reeds sold as identical by the manufacturer, single-reed players such as saxophonists and clarinettists are able to perceive a difference and conclude on the quality of the reed. The ultimate goal is to understand what qualifies a reed as good or bad based on its mechanical characteristics. While Amélie Gaillard's thesis links physical and perceptual studies, my contribution aims to provide additional information through the use of artificial intelligence models. The latter will aim to link audio descriptors measured with an artificial musician and mechanical indicators obtained from stiffness bench measurements. 
\vfill

\clearpage
%-------------Table content--------------------
\tableofcontents
\fancyhead[C]{Table des matières}

\clearpage
%-------------Table content--------------------
\listoffigures
\listoftables
\fancyhead[C]{Tables des figures et des tableaux}
\clearpage


%-------------Remerciement----------------------
\fancyhead[C]{Remerciement}
\begin{center}
\nonumsection{Remerciement}
\end{center}
\vfill
% une quatrième page avec les remerciements. Ils permettent de situer les personnes avec
% lesquelles vous avez travaillé ou échangé pendant le stage ;

Je souhaite dans un premier temps remercier mes encadrants et encadrantes. Ce stage inter-laboratoire implique un encadrement par le Laboratoire d'Acoustique de l'Université du Mans (LAUM) et du Laboratoire d'Informatique de l'Université du Mans (LIUM), respectivement représentés par Bruno Gazengel et Marie Tahon. Je les remercie pour cette association qui a rendu mon stage si complet et diversifié. Ils et elles ont su orienter mes axes de travail et adapter leurs demandes de façon pertinente afin que mon stage soit intéressant et motivant.

Je remercie chaleureusement Amélie Gaillard. Elle a toujours sur trouver du temps pour m'épauler dans mon travail. Les mots justes et des conseils précieux. Beau projet de thèse et j'espère par mon travail avoir permis d'étayer certains aspects de son travail sur le sujet. Je lui souhaite une bonne continuation pour la dernière année de sa thèse.

Un grand merci à Emmanuel Brasseur, INTITULE DU POSTE, qui a été d'une aide précieuse dans la mise en place des bancs de mesures et leur automatisation. Toujours un regard pertinent pour me permettre d'ajuster mes méthodes, des conseils constructifs. Merci pour le temps et la patience mis au service du stage. Un support téchnique sans lequel rien n'aurait vraiment pu prendre forme et efficaces de ouf en plus, reactivité.
   
\vfill


\clearpage
\pagenumbering{arabic} % attention on veut des numéros de page maintenant
\setcounter{page}{1}
%-------------Intro------------------------------
\nonumsection{Introduction}
\fancyhead[C]{Introduction}
% une cinquième page commençant par l'introduction qui définit le contexte dans lequel se
% situe le stage, les objectifs et le plan du document. Cette page est numérotée 1

\vfill

Une anche, c'est ça cane de Provence, processus de création long. défini par une force, une taille et une marque. 
Pour les musiciens, c'est super important et trouverbonne anche, au sens de se sentir à l'aise de l'utiliser pour un jouer u concert en toute confiance, ca peut vite être un calvaire. Bien que les fabricant d'anche vendent des paquets d'anches théoriquement identique, on se rend vite compte en tant que musicien que ce n'est pas le cas dans la réalité. Sur un paquet de 10, 3 anches seronts jouables en concert, 3 seront injouable et le reste seront de qualité moyenne. Si différentes études existe déjà, pas de consensius très concluant. 
Des études perceptives sont actuellement mise en relation avec les caractéristiques physiques des anches  par Amélie Gaillard à travers sa thèse prenant pour problématique "". L'objectifs de mon stage est de s'écarter du caractère perceptif pour analyser les descipteurs audio et essayer d'établir une corrélation avec les indicateurs mécaniques de raideurs. Pour ce faire, on réalisera des mesures à l'aide d'un musicien artificiel en dépression (MAD) apportant de la robustesse dans les mesures et le positionnement de capteurs supplémentaires par rapport aux conditions de jeu d'un instrumentiste.


\vfill


%--------------------------------------------------------
%-------------DEVELOPPEMENT------------------------------
%--------------------------------------------------------
%%%%%%%%%%%%%%%%%%%%%%%%%%%%%%%%%%%%%%%%%%%%%%%%%%%%%%%%%
% structuré en plusieurs parties de taille équilibrée (quatre ou cinq parties
% maximum) dont : présentation de l’entreprise, description des travaux effectués. La structure
% ne doit pas être une vue chronologique du stage mais plutôt une vue par thème. On décrira
% précisément les méthodes utilisées à l'aide si nécessaire de schémas simples et clairs. Chaque
% choix devra être justifié et chaque résultat commenté. Il faut indiquer nettement les résultats
% qui ont été personnellement obtenus. Pour le stage de troisième année, la structure du rapport
% doit se conformer strictement au modèle imposé 
%%%%%%%%%%%%%%%%%%%%%%%%%%%%%%%%%%%%%%%%%%%%%%%%%%%%%%%%%

\clearpage
%-------------Partie 1 : Contexte du stage----------------
\section{Contexte du stage}
\fancyhead[C]{Contexte du stage}
\subsection{Présentation des laboratoires et des équipes}

    Le Mans Université présente plusieurs unités de formations, permettant la mise en place de collaboration entre les laboratoires. C'est dans cette interdisciplinarité qu'a été imaginé le sujet : entre acoustique et informatique. Par conséquent, mon stage est dit interlaboratoire et s'articule entre le Laboratoire d'Acoustique de l'Université du Mans (\href{https://laum.univ-lemans.fr/fr/index.html}{LAUM}) et le Laboratoire d'Informatique de l'Université du Mans (\href{https://lium.univ-lemans.fr/}{LIUM}). Un tel stage implique un double encadrement qui permet d'apporter des visions complémentaires et de segmenter des tâches dans le dérouler du projet. \\


    Le LAUM, d'une part, est une Unité Mixte de Recherche, c'est à une structure administrative réunissant des chercheurs dans le cadre d'un partenariat entre le CNRS et l'université du Mans. Au total, les enseignants.es-chercheurs.euses, chercheurs.euses, doctorants.es et post-doctorants.es, ingénieurs.es et techniciens.nes qui y travaillent, représentent plus 150 personnes. Dirigé par Olivier DAZEL, accompagné de Pascal PICART et Simon FELIX, le laboratoire est spécialisé dans l'étude des sons audibles ou non et des vibrations. Les domaines d'étude se répartissent en trois équipes de recherche : Matériaux, Transducteurs, Guides \& Structures. L'équipe réduite spécialisée dans l'étude de la physique des instruments de musique fait partie de l'axe des Guides \& Structures. \\

    
    Le LIUM est, quant à lui, créé dans les années 90, et dénombre à ce jour une cinquantaine de membres parmi lesquels 25 enseignants.es-chercheurs.euses. Les travaux qui y prennent place son ciblé sur l'informatique et plus spécifiquement l’Ingénierie des Environnements Informatiques pour l’Apprentissage Humain (IEIAH) et le traitement automatique de la langue (LST). Le laboratoire est localisé entre deux sites : le bâtiment IC2 au Mans, et le site de Laval dans les bâtiments de l’IUT. La direction actuelle est constituée de Sébastien George (directeur) et son adjoint, Sylvain Meignier.


\subsection{Objectifs et enjeux du sujet}

        SUpport d'une thèse en cours, équipe de recherhce sur le sujet déjà établie 
        Apport d'un autre point de vue, expérimental et sans garanti de résultats, nouvelle piste qui se veut complémentaire et travaile qui permettra de déterminer si on creuse ou pas
        Comment s'affranchir de la partie perceptive et plutôt utiliser des descripteurs audios extrait de signaux répétables par Musicien artificiel : implémenter des indicateurs mécaniques sur l'anche dans sa globalité ou axé sur des particularités de la coupe, et observer des potentielles corrélations avec des descripteurs audios choisis


\clearpage
%-------------Partie 2 : Etat de l'art-----------------
\section{État de l'art}
\fancyhead[C]{État de l'art}
\subsection{Etat de l'existant}

    % Pour un musicien ou une musicienne jouant d'un instrument à anche simple comme le saxophone et la clarinette, le choix de l'anche à une grande importance dans la qualité sonore du jeu.
    - Fonctionnement d'un instrument à anche simple
    - importance de l'anche dans le jeu d'un musicien
    - caractérisation d'une anche à la production (coupe, force, matériaux, marque)
    - paramètre perceptif (proche à l'instrumentiste, plus facile à percevoir des différences pour le musicien que l'auditoire, car plus de transmission)

    faire un plan détaillé pour l'instant (voir rapport d'avancemeent pour CSI d'Amélie)


    

\clearpage
%-------------Partie 3 : Matériel et méthode-----------------
\section{Matériel et méthode}
\fancyhead[C]{Matériel et méthode}
% 3.1. Banc Statique
%     - Schéma de principe/ photo légendée
%     - Capteur de force : calibration, extraction des données
%     - Caméra : déplacement
%     - automatisation, contrôle de position pour répétabilité, répétabilité, validation 
%     - fromat des données : profil raideur + déformation du profil
% 3.2. MAD Statique - raideur d'anche couplée avec le bec / en situation de jeu
%     - Schéma de principe / photo légendée (capteurs)
%     - lèvre, caméra
%     - données : raideur globale du profil, canal d'anche,
% 3.3. MAD Dynamique
%     - ajout d'un dispositif à partir du mad statique --> création d'une dépression pour mise en vibration de l'anche = jeu 
%     - données : foce, pression int et ext, déplacement 

\subsection{Profile de raideur d'anche statique}


\subsection{Raideur d'anche couplée au bec/en situation de jeu}

\subsection{MAD Dynamique}



\clearpage
%-------------Partie 4 : Mesures-----------------
\section{Mesures}
\fancyhead[C]{Mesures}
% 4.1 choix des échantillons, choix des paramètres d'acquisition, chaines de mesures, protocole (point sur largeur d'anche...)
% 4.2 description des protocoles, temps de mesures... --> pourl'instant faire un planning prévisionnel

\subsection{Protocoles de mesures, choix des échantillons étudiés}

\subsection{Réalisation des mesures}


\clearpage
%-------------Partie 5 : Analyse des données-----------------
\section{Analyse des données}
\fancyhead[C]{Analyse des données}
% 5.1. Indicateurs mécaniques 
%     5.1.1. Profil de raideur 
%     5.1.2. déplacement profile
%     5.1.3. Raideur global et enroulement --> courbe paramétrique
% 5.2. Descripteur audio
%     matrice 2*4 en fonction de Pm
% 5.3. Modèle d'IA ...

\subsection{Indicateur mécanique}


\subsection{Déscripteurs audio}


\subsection{Modèle de traitement par Intelligence Artifielle}


\clearpage
%-------------Partie 5 : Résultats et conclusions-----------------
\section{Résultats et conclusions}
\fancyhead[C]{Résultats et conclusions}
\input{section/10.Résultats et conclusions}


%-----------------------------------

\clearpage
%------------Conclusion-------------
\nonumsection{Conclusion}
\fancyhead[C]{Conclusion}
% devant synthétiser le travail effectué en soulignant les apports réalisés du
% point de vue technique et personnel. Un regard critique peut être apporté en faisant preuve de
% suffisamment de recul. Il est important aussi de préciser les perspectives envisagées ;

\vfill

    En conclusion, la combinaison de modèles numériques et de mesures expérimentales dans une cabine alpha offre une approche complète pour comprendre et améliorer les performances acoustiques des matériaux. Ainsi, dans ce rapport, on a pu mettre en évidence l'importance de la loi de masse sur le comportement des pertes par transmission. On a également démontré le rôle du module de Young des matériaux fibreux et des autres paramètres qui permettaient d'adapter notre simulation pour correspondre au mieux aux mesures du CTTM. Enfin, on a pu s'intéresser à la stratégie de séparation par des couches d'air pour une insonorisation efficace et adaptée aux besoins spécifiques de diverses applications acoustiques. Toutefois, de manière générale, pour atteindre un indice d'affaiblissement $R_{w}$ de 50 dB, la seule solution viable reste de rendre le panneau plus massique.

    \gls{latex}

\vfill

\clearpage
%-----------Biblio--------------
\fancyhead[c]{Bibliogrpahie}
\printbibliography

\clearpage
%-----------Glossaire--------------
\fancyhead[C]{Glossaire}


\newglossaryentry{latex}
{
    name=LaTeX,
    description={Un système de composition de documents très utilisé dans le domaine académique}
}



\printglossaries

\clearpage
%-----------Annexe--------------
\nonumsection{Annexe}
\fancyhead[c]{Annexe}
\input{section/13.Annexe}

\end{document}


