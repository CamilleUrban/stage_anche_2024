Plan :


1. Contexte du stage 
    1.1. Stage en Laboratoire - interlaboratoire
    1.2. justification du sujet de stage : équipe qui travaille déjà sur les anches, Amélie en thèse, Intéret de mon stage, complément de la thèse, Objectifs et enjeux du sujet
    
2. État de l'art 
     2.1. Anches
     faire un plan détaillé pour l'instant (voir rapport d'avancemeent pour CSI d'Amélie)


3. Matériel et méthode
    3.1. Banc Statique
        - Schéma de principe/ photo légendée
        - Capteur de force : calibration, extraction des données
        - Caméra : déplacement
        - automatisation, contrôle de position pour répétabilité, répétabilité, validation 
        - fromat des données : profil raideur + déformation du profil
    3.2. MAD Statique - raideur d'anche couplée avec le bec / en situation de jeu
        - Schéma de principe / photo légendée (capteurs)
        - lèvre, caméra
        - données : raideur globale du profil, canal d'anche,
    3.3. MAD Dynamique
        - ajout d'un dispositif à partir du mad statique --> création d'une dépression pour mise en vibration de l'anche = jeu 
        - données : foce, pression int et ext, déplacement 

4. Mesures 
    4.1 choix des échantillons, choix des paramètres d'acquisition, chaines de mesures, protocole (point sur largeur d'anche...)
    4.2 description des protocoles, temps de mesures... --> pourl'instant faire un planning prévisionnel

5. Analyse des données -> format des données pour traitement IA
    5.1. Indicateurs mécaniques 
        5.1.1. Profil de raideur 
        5.1.2. déplacement profile
        5.1.3. Raideur global et enroulement --> courbe paramétrique
    5.2. Descripteur audio
        matrice 2*4 en fonction de Pm
    5.3. Modèle d'IA ...
    
6. Résultats et conclusions