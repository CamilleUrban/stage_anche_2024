% une cinquième page commençant par l'introduction qui définit le contexte dans lequel se
% situe le stage, les objectifs et le plan du document. Cette page est numérotée 1

\vfill

Une anche, c'est ça cane de Provence, processus de création long. défini par une force, une taille et une marque. 
Pour les musiciens, c'est super important et trouverbonne anche, au sens de se sentir à l'aise de l'utiliser pour un jouer u concert en toute confiance, ca peut vite être un calvaire. Bien que les fabricant d'anche vendent des paquets d'anches théoriquement identique, on se rend vite compte en tant que musicien que ce n'est pas le cas dans la réalité. Sur un paquet de 10, 3 anches seronts jouables en concert, 3 seront injouable et le reste seront de qualité moyenne. Si différentes études existe déjà, pas de consensius très concluant. 
Des études perceptives sont actuellement mise en relation avec les caractéristiques physiques des anches  par Amélie Gaillard à travers sa thèse prenant pour problématique "". L'objectifs de mon stage est de s'écarter du caractère perceptif pour analyser les descipteurs audio et essayer d'établir une corrélation avec les indicateurs mécaniques de raideurs. Pour ce faire, on réalisera des mesures à l'aide d'un musicien artificiel en dépression (MAD) apportant de la robustesse dans les mesures et le positionnement de capteurs supplémentaires par rapport aux conditions de jeu d'un instrumentiste.


\vfill
