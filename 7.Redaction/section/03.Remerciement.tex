% une quatrième page avec les remerciements. Ils permettent de situer les personnes avec
% lesquelles vous avez travaillé ou échangé pendant le stage ;

Je souhaite dans un premier temps remercier mes encadrants et encadrantes. Ce stage inter-laboratoire implique un encadrement par le Laboratoire d'Acoustique de l'Université du Mans (LAUM) et du Laboratoire d'Informatique de l'Université du Mans (LIUM), respectivement représentés par Bruno Gazengel et Marie Tahon. Je les remercie pour cette association qui a rendu mon stage si complet et diversifié. Ils et elles ont su orienter mes axes de travail et adapter leurs demandes de façon pertinente afin que mon stage soit intéressant et motivant.

Je remercie chaleureusement Amélie Gaillard. Elle a toujours sur trouver du temps pour m'épauler dans mon travail. Les mots justes et des conseils précieux. Beau projet de thèse et j'espère par mon travail avoir permis d'étayer certains aspects de son travail sur le sujet. Je lui souhaite une bonne continuation pour la dernière année de sa thèse.

Un grand merci à Emmanuel Brasseur, INTITULE DU POSTE, qui a été d'une aide précieuse dans la mise en place des bancs de mesures et leur automatisation. Toujours un regard pertinent pour me permettre d'ajuster mes méthodes, des conseils constructifs. Merci pour le temps et la patience mis au service du stage. Un support téchnique sans lequel rien n'aurait vraiment pu prendre forme et efficaces de ouf en plus, reactivité.
