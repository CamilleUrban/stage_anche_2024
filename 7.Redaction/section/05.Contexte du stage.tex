\subsection{Présentation des laboratoires et des équipes}

    Le Mans Université présente plusieurs unités de formations, permettant la mise en place de collaboration entre les laboratoires. C'est dans cette interdisciplinarité qu'a été imaginé le sujet : entre acoustique et informatique. Par conséquent, mon stage est dit interlaboratoire et s'articule entre le Laboratoire d'Acoustique de l'Université du Mans (\href{https://laum.univ-lemans.fr/fr/index.html}{LAUM}) et le Laboratoire d'Informatique de l'Université du Mans (\href{https://lium.univ-lemans.fr/}{LIUM}). Un tel stage implique un double encadrement qui permet d'apporter des visions complémentaires et de segmenter des tâches dans le dérouler du projet. \\


    Le LAUM, d'une part, est une Unité Mixte de Recherche, c'est à une structure administrative réunissant des chercheurs dans le cadre d'un partenariat entre le CNRS et l'université du Mans. Au total, les enseignants.es-chercheurs.euses, chercheurs.euses, doctorants.es et post-doctorants.es, ingénieurs.es et techniciens.nes qui y travaillent, représentent plus 150 personnes. Dirigé par Olivier DAZEL, accompagné de Pascal PICART et Simon FELIX, le laboratoire est spécialisé dans l'étude des sons audibles ou non et des vibrations. Les domaines d'étude se répartissent en trois équipes de recherche : Matériaux, Transducteurs, Guides \& Structures. L'équipe réduite spécialisée dans l'étude de la physique des instruments de musique fait partie de l'axe des Guides \& Structures. \\

    
    Le LIUM est, quant à lui, créé dans les années 90, et dénombre à ce jour une cinquantaine de membres parmi lesquels 25 enseignants.es-chercheurs.euses. Les travaux qui y prennent place son ciblé sur l'informatique et plus spécifiquement l’Ingénierie des Environnements Informatiques pour l’Apprentissage Humain (IEIAH) et le traitement automatique de la langue (LST). Le laboratoire est localisé entre deux sites : le bâtiment IC2 au Mans, et le site de Laval dans les bâtiments de l’IUT. La direction actuelle est constituée de Sébastien George (directeur) et son adjoint, Sylvain Meignier.


\subsection{Objectifs et enjeux du sujet}

        SUpport d'une thèse en cours, équipe de recherhce sur le sujet déjà établie 
        Apport d'un autre point de vue, expérimental et sans garanti de résultats, nouvelle piste qui se veut complémentaire et travaile qui permettra de déterminer si on creuse ou pas
        Comment s'affranchir de la partie perceptive et plutôt utiliser des descripteurs audios extrait de signaux répétables par Musicien artificiel : implémenter des indicateurs mécaniques sur l'anche dans sa globalité ou axé sur des particularités de la coupe, et observer des potentielles corrélations avec des descripteurs audios choisis