% Résumé
% un résumé (<160 mots) posant BRIEVEMENT le contexte du
% stage, la façon dont il a été abordé et les principaux résultats obtenus. De plus, vous ajouterez à la suite les mots clés renseignés dans la fiche du stage

Ce stage s'inscrit dans le cadre d'une thèse en cours intitulée ... et menée par Amélie Gaillard. L'anche est un petit bout de roseau taillé qui est mis en vibration par le souffle du musicien, créant alors un son. Toute fois, pour deux anches vendues comme identiques par le fabricant, des musiciens à anches simples, comme les saxophonistes et les clarinettistes, sont capables de percevoir une différence et de conclure sur la qualité de l'anche. L'objectif final est de comprendre ce qui permet de qualifier une anche comme bonne ou mauvaise en se basant sur les caractéristiques mécaniques de cette dernière. Si la thèse d'Amélie Gaillard vient mettre en lien des études physiques et perceptives, ma contribution a pour but d'apporter des informations complémentaires via l'utilisation de modèles intelligence artificielle. Ces derniers viseront à mettre en lien des descripteurs audio mesurés à l'aide d'un musicien artificiel et des indicateurs mécaniques obtenus par des mesures sur bancs de raideurs. 
