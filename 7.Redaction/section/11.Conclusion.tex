% devant synthétiser le travail effectué en soulignant les apports réalisés du
% point de vue technique et personnel. Un regard critique peut être apporté en faisant preuve de
% suffisamment de recul. Il est important aussi de préciser les perspectives envisagées ;

\vfill

    En conclusion, la combinaison de modèles numériques et de mesures expérimentales dans une cabine alpha offre une approche complète pour comprendre et améliorer les performances acoustiques des matériaux. Ainsi, dans ce rapport, on a pu mettre en évidence l'importance de la loi de masse sur le comportement des pertes par transmission. On a également démontré le rôle du module de Young des matériaux fibreux et des autres paramètres qui permettaient d'adapter notre simulation pour correspondre au mieux aux mesures du CTTM. Enfin, on a pu s'intéresser à la stratégie de séparation par des couches d'air pour une insonorisation efficace et adaptée aux besoins spécifiques de diverses applications acoustiques. Toutefois, de manière générale, pour atteindre un indice d'affaiblissement $R_{w}$ de 50 dB, la seule solution viable reste de rendre le panneau plus massique.

    \gls{latex}

\vfill